\documentclass{sigchi}
\pagenumbering{arabic}

% Load basic packages
\usepackage{balance}       % to better equalize the last page
\usepackage{graphics}      % for EPS, load graphicx instead 
\usepackage[T1]{fontenc}   % for umlauts and other diaeresis
\usepackage{txfonts}
\usepackage{mathptmx}
\usepackage[pdflang={en-US},pdftex]{hyperref}
\usepackage{color}
\usepackage{booktabs}
\usepackage{textcomp}

% Some optional stuff you might like/need.
\usepackage{microtype}        % Improved Tracking and Kerning
% \usepackage[all]{hypcap}    % Fixes bug in hyperref caption linking
\usepackage{ccicons}          % Cite your images correctly!
% \usepackage[utf8]{inputenc} % for a UTF8 editor only

% If you want to use todo notes, marginpars etc. during creation of
% your draft document, you have to enable the "chi_draft" option for
% the document class. To do this, change the very first line to:
% "\documentclass[chi_draft]{sigchi}". You can then place todo notes
% by using the "\todo{...}"  command. Make sure to disable the draft
% option again before submitting your final document.
\usepackage{todonotes}

% Paper metadata
\def\plaintitle{CS197 Assignment 2: Related Work}
\def\plainauthor{}
\def\emptyauthor{}
\def\plainkeywords{}
\def\plaingeneralterms{}

% llt: Define a global style for URLs, rather that the default one
\makeatletter
\def\url@leostyle{%
  \@ifundefined{selectfont}{
    \def\UrlFont{\sf}
  }{
    \def\UrlFont{\small\bf\ttfamily}
  }}
\makeatother
\urlstyle{leo}

% To make various LaTeX processors do the right thing with page size.
\def\pprw{8.5in}
\def\pprh{11in}
\special{papersize=\pprw,\pprh}
\setlength{\paperwidth}{\pprw}
\setlength{\paperheight}{\pprh}
\setlength{\pdfpagewidth}{\pprw}
\setlength{\pdfpageheight}{\pprh}

% Make sure hyperref comes last of your loaded packages, to give it a
% fighting chance of not being over-written, since its job is to
% redefine many LaTeX commands.
\definecolor{linkColor}{RGB}{6,125,233}
\hypersetup{%
  pdftitle={\plaintitle},
% Use \plainauthor for final version.
%  pdfauthor={\plainauthor},
  pdfauthor={\emptyauthor},
  pdfkeywords={\plainkeywords},
  pdfdisplaydoctitle=true, % For Accessibility
  bookmarksnumbered,
  pdfstartview={FitH},
  colorlinks,
  citecolor=black,
  filecolor=black,
  linkcolor=black,
  urlcolor=linkColor,
  breaklinks=true,
  hypertexnames=false
}

%% remove the copyright box
\makeatletter
\def\@copyrightspace{\relax}
\makeatother

% create a shortcut to typeset table headings
% \newcommand\tabhead[1]{\small\textbf{#1}}

% End of preamble. Here it comes the document.
\begin{document}

\title{\plaintitle}

\numberofauthors{3}
\author{%
  \alignauthor{Project Member 1\\
    \email{Stanford Email}}\\
  \alignauthor{Project Member 2\\
    \email{Stanford Email}}\\
  \alignauthor{Project Member 3\\
    \email{Stanford Email}}\\
  }

\maketitle

%%%%%%%%%%%%%%%%%%%% TODO: Fill in this section
\section{Related Work}
Fill in this section with your Related Work content for Assignment 2. Delete the below section before compiling and submitting your single group PDF through Canvas. Here's an example of a citation using BibTeX~\cite{acm_categories}. More info about this template below.


%%%%%%%%%%%%%%%%%%%% DELETE THE BELOW SECTIONS %%%%%%%%%%%%%%%%%%%%
% The below sections contain useful info for starting out with LaTeX.
% Before submitting, delete everything between the two comment lines % with many %'s (i.e., from the above line to the one way down below)
\vfill\break
\section{Delete the below sections before submitting on Canvas}

\section{CS197: Overview}
This format is based on the SIGCHI Conference Proceedings Format, which you can find at this URL \url{https://www.overleaf.com/latex/templates/chi2020-proceedings/qtdvrwbtqxww}. We've included portions of the original SIGCHI template text that provide guidance on the assignment.

\subsection{Using Overleaf}
\textbf{Import a template.}  Download the template zip from the CS197 website. Do not open/extract these files. On the Overleaf website (\url{https://www.overleaf.com/}) (register with your stanford.edu email), go to "New Project" -> "Upload Project." Upload the zip file you downloaded. Then Overleaf will automatically extract this zip by creating a new project.

\textbf{Share a project.} If you are the \textit{owner} of the project, go to "Share" and add your project group members by Stanford email. If you are \textit{invited} to a project, find the Overleaf project invitation in your email (the title should be something like "\dots shared by\dots") and click on "View Project." Note that you must click on this invitation link before you can access the editable Overleaf page.
    
\textbf{Export a project.} In the Overleaf editor for a project, click the Download icon next to the Recompile button. This generates and downloads a PDF that you can view on your local computer and upload to Canvas.

\textbf{Update authors.} Search for this portion of the early part of the LaTeX file:

\begin{verbatim}
  \author{%
    \alignauthor{Project Member 1\\
      \email{Stanford Email}}\\
    ...
  }
\end{verbatim}

and update as needed.


\textbf{Update title.} Search for the below line in the early part of this LaTeX file and edit what's in the last set of curly braces: \verb!\def\plaintitle{CS197 Assignment 2: ...}!

\textbf{Include citations via BibTeX.} Most citations will have BibTeX metadata readily available through a few clicks (e.g., Google Scholar: \url{https://libguides.usask.ca/c.php?g=218034&p=1445751}). Copy the BibTeX metadata into the \texttt{bibliography.bib} file located in this project. Then, you can cite by using \verb!\cite{X}!, where \texttt{X} is the reference name. For an example, see the Reference and Citations subsection below. Note that you can change the reference name to be something more meaningful, but you shouldn't change the other fields like title, author, year, etc.


\section{SIGCHI Format: Typeset Text}
The styles contained in this document have been modified from the
default styles to reflect ACM formatting conventions. For example,
content paragraphs like this one are formatted using the Normal style.

\LaTeX\ sometimes will create overfull lines that extend into columns.
To attempt to combat this, the \texttt{.cls} file has a command,
\texttt{{\textbackslash}sloppy}, that essentially asks \LaTeX\ to
prefer underfull lines with extra whitespace.  For more details on
this, and info on how to control it more finely, check out
{\url{http://www.economics.utoronto.ca/osborne/latex/PMAKEUP.HTM}}.

\subsection{References and Citations}

Use a numbered list of references at the end of the article, ordered
alphabetically by last name of first author, and referenced by numbers
in
brackets~\cite{acm_categories,ethics,Klemmer:2002:WSC:503376.503378}.
Your references should be published materials accessible to the
public. Internal technical reports may be cited only if they are
easily accessible (i.e., you provide the address for obtaining the
report within your citation) and may be obtained by any reader for a
nominal fee. Proprietary information may not be cited. Private
communications should be acknowledged in the main text, not referenced
(e.g., ``[Borriello, personal communication]'').

References should be in ACM citation format:
\url{http://acm.org/publications/submissions/latex_style}. This
includes citations to internet
resources~\cite{acm_categories,cavender:writing,CHINOSAUR:venue,psy:gangnam}
according to ACM format, although it is often appropriate to include
URLs directly in the text, as above.


% Use a numbered list of references at the end of the article, ordered
% alphabetically by first author, and referenced by numbers in
% brackets~\cite{ethics, Klemmer:2002:WSC:503376.503378,
%   Mather:2000:MUT, Zellweger:2001:FAO:504216.504224}. For papers from
% conference proceedings, include the title of the paper and an
% abbreviated name of the conference (e.g., for Interact 2003
% proceedings, use \textit{Proc. Interact 2003}). Do not include the
% location of the conference or the exact date; do include the page
% numbers if available. See the examples of citations at the end of this
% document. Within this template file, use the \texttt{References} style
% for the text of your citation.

% Your references should be published materials accessible to the
% public.  Internal technical reports may be cited only if they are
% easily accessible (i.e., you provide the address for obtaining the
% report within your citation) and may be obtained by any reader for a
% nominal fee.  Proprietary information may not be cited. Private
% communications should be acknowledged in the main text, not referenced
% (e.g., ``[Robertson, personal communication]'').

\section{SIGCHI Format: Sections}

The heading of a section should be in Helvetica or Arial 9-point bold,
all in capitals. Sections should \textit{not} be numbered.

\subsection{Subsections}

Headings of subsections should be in Helvetica or Arial 9-point bold
with initial letters capitalized.  For sub-sections and
sub-subsections, a word like \emph{the} or \emph{of} is not
capitalized unless it is the first word of the heading.

\subsubsection{Sub-subsections}

Headings for sub-subsections should be in Helvetica or Arial 9-point
italic with initial letters capitalized.  Standard
\texttt{{\textbackslash}section}, \texttt{{\textbackslash}subsection},
and \texttt{{\textbackslash}subsubsection} commands will work fine in
this template.

\section{Quotations}
Quotations may be italicized when \textit{``placed inline''}.

\begin{quote}
Longer quotes, when placed in their own paragraph, need not be
italicized or in quotation marks when indented.  
\end{quote}

\section{Language, Style, and Content}

The written and spoken language of SIGCHI is English. Spelling and
punctuation may use any dialect of English (e.g., British, Canadian,
US, etc.) provided this is done consis- tently. Hyphenation is
optional. To ensure suitability for an international audience, please
pay attention to the following:

\begin{itemize}
\item Write in a straightforward style.
\item Try to avoid long or complex sentence structures.
\item Use common and basic vocabulary (e.g., use the word ``unusual'' rather than the word ``arcane''.
\item Briefly define or explain all technical terms that may be
  unfamiliar to readers.
\item Explain all acronyms the first time they are used in your
  text---e.g., ``Digital Signal Processing (DSP)''.
\item Explain local references (e.g., not everyone knows all city
  names in a particular country).
\item Explain ``insider'' comments. Ensure that your whole audience
  understands any reference whose meaning you do not describe (e.g.,
  do not assume that everyone has used a Macintosh or a particular
  application).
\item Explain colloquial language and puns. Understanding phrases like
  ``red herring'' may require a local knowledge of English.  Humor and
  irony are difficult to translate.
\item Use unambiguous forms for culturally localized concepts, such as
  times, dates, currencies, and numbers (e.g., ``1--5--97'' or
  ``5/1/97'' may mean 5 January or 1 May, and ``seven o'clock'' may
  mean 7:00 am or 19:00). For currencies, indicate equivalences:
  ``Participants were paid {\fontfamily{txr}\selectfont \textwon}
  25,000, or roughly US \$22.''
\item Be careful with the use of gender-specific pronouns (he, she)
  and other gendered words (chairman, manpower, man-months). Use
  inclusive language that is gender-neutral (e.g., she or he, they,
  s/he, chair, staff, staff-hours, person-years). See the
  \textit{Guidelines for Bias-Free Writing} for further advice and
  examples regarding gender and other personal
  attributes~\cite{Schwartz:1995:GBF}. Be particularly aware of
  considerations around writing about people with disabilities.
\item If possible, use the full (extended) alphabetic character set
  for names of persons, institutions, and places (e.g.,
  Gr{\o}nb{\ae}k, Lafreni\'ere, S\'anchez, Nguy{\~{\^{e}}}n,
  Universit{\"a}t, Wei{\ss}enbach, Z{\"u}llighoven, \r{A}rhus, etc.).
  These characters are already included in most versions and variants
  of Times, Helvetica, and Arial fonts.
\end{itemize}

\section{Page Numbering, Headers and Footers}
Your final submission should not contain footer or header information
at the top or bottom of each page. Specifically, your final submission
should not include page numbers. Initial submissions may include page
numbers, but these must be removed for camera-ready. Page numbers will
be added to the PDF when the proceedings are assembled.



% Balancing columns in a ref list is a bit of a pain because you
% either use a hack like flushend or balance, or manually insert
% a column break.  http://www.tex.ac.uk/cgi-bin/texfaq2html?label=balance
% multicols doesn't work because we're already in two-column mode,
% and flushend isn't awesome, so I choose balance.  See this
% for more info: http://cs.brown.edu/system/software/latex/doc/balance.pdf
%
% Note that in a perfect world balance wants to be in the first
% column of the last page.
%
% If balance doesn't work for you, you can remove that and
% hard-code a column break into the bbl file right before you
% submit:
%
% http://stackoverflow.com/questions/2149854/how-to-manually-equalize-columns-
% in-an-ieee-paper-if-using-bibtex
%
% Or, just remove \balance and give up on balancing the last page.
%
\balance{}

\section{SIGCHI Format: References Format}
(CS197 Note: This section is included for your information. Including the two lines \begin{verbatim}
   \bibliographystyle{SIGCHI-Reference-Format}
   \bibliography{bibliography}
\end{verbatim}

in your \LaTeX document will auto-format the bibliography according to these standards.)

Your references should be published materials accessible to the
public. Internal technical reports may be cited only if they are
easily accessible and may be obtained by any reader for a nominal
fee. Proprietary information may not be cited. Private communications
should be acknowledged in the main text, not referenced (e.g.,
[Golovchinsky, personal communication]). References must be the same
font size as other body text. References should be in alphabetical
order by last name of first author. Use a numbered list of references
at the end of the article, ordered alphabetically by last name of
first author, and referenced by numbers in brackets. For papers from
conference proceedings, include the title of the paper and the name of
the conference. Do not include the location of the conference or the
exact date; do include the page numbers if available. 

References should be in ACM citation format:
\url{http://www.acm.org/publications/submissions/latex_style}.  This
includes citations to Internet
resources~\cite{CHINOSAUR:venue,cavender:writing,psy:gangnam}
according to ACM format, although it is often appropriate to include
URLs directly in the text, as above. Example reference formatting for
individual journal articles~\cite{ethics}, articles in conference
proceedings~\cite{Klemmer:2002:WSC:503376.503378},
books~\cite{Schwartz:1995:GBF}, theses~\cite{sutherland:sketchpad},
book chapters~\cite{winner:politics}, an entire journal
issue~\cite{kaye:puc},
websites~\cite{acm_categories,cavender:writing},
tweets~\cite{CHINOSAUR:venue}, patents~\cite{heilig:sensorama}, 
games~\cite{supermetroid:snes}, and
online videos~\cite{psy:gangnam} is given here.  See the examples of
citations at the end of this document and in the accompanying
\texttt{BibTeX} document. This formatting is a edited version of the
format automatically generated by the ACM Digital Library
(\url{http://dl.acm.org}) as ``ACM Ref.'' DOI and/or URL links are
optional but encouraged as are full first names. Note that the
Hyperlink style used throughout this document uses blue links;
however, URLs in the references section may optionally appear in
black.


%%%%%%%%%%%%%%%%%%%% DELETE THE ABOVE SECTIONS %%%%%%%%%%%%%%%%%%%%

% BALANCE COLUMNS
\balance{}

% REFERENCES FORMAT
% References must be the same font size as other body text.
\bibliographystyle{SIGCHI-Reference-Format}
\bibliography{bibliography}

\end{document}

%%% Local Variables:
%%% mode: latex
%%% TeX-master: t
%%% End:
