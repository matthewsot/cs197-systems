%% For double-blind review submission, w/o CCS and ACM Reference (max submission space)
% \documentclass[acmsmall,review,screen,anonymous]{acmart}\settopmatter{printfolios=true,printccs=false,printacmref=false}
%% For double-blind review submission, w/ CCS and ACM Reference
%\documentclass[acmsmall,review,anonymous]{acmart}\settopmatter{printfolios=true}
%% For single-blind review submission, w/o CCS and ACM Reference (max submission space)
%\documentclass[acmsmall,review]{acmart}\settopmatter{printfolios=true,printccs=false,printacmref=false}
%% For single-blind review submission, w/ CCS and ACM Reference
%\documentclass[acmsmall,review]{acmart}\settopmatter{printfolios=true}
%% For final camera-ready submission, w/ required CCS and ACM Reference
\documentclass[acmsmall]{acmart}\settopmatter{}


%% Journal information
%% Supplied to authors by publisher for camera-ready submission;
%% use defaults for review submission.
\acmJournal{PACMPL}
\acmVolume{1}
\acmNumber{CONF} % CONF = POPL or ICFP or OOPSLA
\acmArticle{1}
\acmYear{2018}
\acmMonth{1}
\acmDOI{} % \acmDOI{10.1145/nnnnnnn.nnnnnnn}
\startPage{1}


%% Copyright information
%% Supplied to authors (based on authors' rights management selection;
%% see authors.acm.org) by publisher for camera-ready submission;
%% use 'none' for review submission.
\setcopyright{none}
%\setcopyright{acmcopyright}
%\setcopyright{acmlicensed}
%\setcopyright{rightsretained}
%\copyrightyear{2018}           %% If different from \acmYear

%% Bibliography style
\bibliographystyle{ACM-Reference-Format}
%% Citation style
%% Note: author/year citations are required for papers published as an
%% issue of PACMPL.
\citestyle{acmauthoryear}   %% For author/year citations


%%%%%%%%%%%%%%%%%%%%%%%%%%%%%%%%%%%%%%%%%%%%%%%%%%%%%%%%%%%%%%%%%%%%%%
%% Note: Authors migrating a paper from PACMPL format to traditional
%% SIGPLAN proceedings format must update the '\documentclass' and
%% topmatter commands above; see 'acmart-sigplanproc-template.tex'.
%%%%%%%%%%%%%%%%%%%%%%%%%%%%%%%%%%%%%%%%%%%%%%%%%%%%%%%%%%%%%%%%%%%%%%


%% Some recommended packages.
\usepackage{booktabs}   %% For formal tables:
                        %% http://ctan.org/pkg/booktabs
\usepackage{subcaption} %% For complex figures with subfigures/subcaptions
                        %% http://ctan.org/pkg/subcaption

\usepackage{listings}

\usepackage{xcolor}
% \newcommand{\meta}[1]{{\color{blue}\textbf{[}#1\textbf{]}}}

\usepackage{tikz}
\usetikzlibrary{arrows}

\usepackage{mathtools}
\DeclarePairedDelimiter\ceil{\lceil}{\rceil}
\DeclarePairedDelimiter\floor{\lfloor}{\rfloor}

\usepackage{prettyref}
\newcommand{\pref}{\prettyref}
\newrefformat{thm}{Theorem~\ref{#1}}
\newrefformat{cor}{Corollary~\ref{#1}}
\newrefformat{lem}{Lemma~\ref{#1}}
\newrefformat{cha}{Chapter~\ref{#1}}
\newrefformat{sec}{Section~\ref{#1}}
\newrefformat{app}{Appendix~\ref{#1}}
\newrefformat{tab}{Table~\ref{#1}}
\newrefformat{fig}{Figure~\ref{#1}}
\newrefformat{alg}{Algorithm~\ref{#1}}
\newrefformat{exa}{Example~\ref{#1}}
\newrefformat{def}{Definition~\ref{#1}}
\newrefformat{li}{Line~\ref{#1}}
\newrefformat{eq}{Equation~\ref{#1}}
\newrefformat{exa}{Example~\ref{#1}}

% Language semantics layout adapted from
% https://dl.acm.org/doi/pdf/10.1145/3371106

\newcommand{\leif}[3]{\left(#1\right)\hspace{1mm}?\hspace{1mm}\left(#2\right)\hspace{1mm}:\hspace{1mm}\left(#3\right)}
\newcommand{\whilevarnothalt}[1]{\mathbf{while}\hspace{1mm}\mathtt{#1}\neq\mathbf{halt}\lkw{do}}
\newcommand{\lkw}[1]{\mathbf{#1}\hspace{1mm}}
\newcommand{\lfn}[1]{\mathbf{#1}}
\newcommand{\lvar}[1]{\mathtt{#1}}
\newcommand{\lval}[1]{\mathbf{val}(#1)}
\newcommand{\lnext}[2]{\mathbf{next}(#1, #2)}
\newcommand{\lhalt}{\mathbf{halt}}
\newcommand{\lnull}{\mathbf{NULL}}
\newcommand{\lisnull}[1]{\mathbf{isnull}(#1)}
\newcommand{\qqquad}{\qquad\quad}

\usepackage{mathtools}

\DeclarePairedDelimiter\abs{\lvert}{\rvert}%
\DeclarePairedDelimiter\norm{\lVert}{\rVert}%

\usepackage[linesnumbered]{algorithm2e}
\SetKwFor{While}{while (}{) $\lbrace$}{$\rbrace$}
\SetKwIF{If}{ElseIf}{Else}{if (}{) $\lbrace$}{$\rbrace$ else if}{$\rbrace$ else $\lbrace$}{$\rbrace$}

% https://stackoverflow.com/questions/4439605/c-source-code-in-latex-document
\lstset{
  language=C,                     % choose the language of the code
  numbers=left,                   % where to put the line-numbers
  stepnumber=1,                   % the step between two line-numbers.        
  numbersep=2pt,                  % how far the line-numbers are from the code
  backgroundcolor=\color{white},  % choose the background color. You must add \usepackage{color}
  showspaces=false,               % show spaces adding particular underscores
  showstringspaces=false,         % underline spaces within strings
  showtabs=false,                 % show tabs within strings adding particular underscores
  tabsize=2,                      % sets default tabsize to 2 spaces
  captionpos=b,                   % sets the caption-position to bottom
  breaklines=true,                % sets automatic line breaking
  breakatwhitespace=true,         % sets if automatic breaks should only happen at whitespace
  % title=\lstname,                 % show the filename of files included with \lstinputlisting;
  basicstyle=\scriptsize,
}

\usetikzlibrary{calc,shapes.multipart,chains,arrows}

\newcommand\tzconnect[3]{
    \draw (#1.south west) -- (#2.north west);
    \draw (#1.south east) -- (#3.north east);
    \draw (#2.south west) -- (#3.south east);
}
\newcommand\tznext[2]{
    \draw[*->] let \p1 = (#1.two), \p2 = (#1.center) in (\x1,\y2) -- (#2);
}
\newcommand\tzstripe[3]{
    \draw[ultra thick,#3] ([yshift=#2]#1.north west) -- ([yshift=#2]#1.north east);
}
\newcommand\tzstriped[3]{
    \draw[ultra thick,#3,-|] ([yshift=#2]#1.north west) -- ([yshift=#2]#1.north east);
}
\newcommand\tzstripel[4]{
    \draw[ultra thick,#4] ([yshift=#3]#1.north west) -- ([yshift=#3]#2.north east);
}
\newcommand\tzstripedl[4]{
    \draw[ultra thick,#4,-|] ([yshift=#3]#1.north west) -- ([yshift=#3]#2.north east);
}
\newcommand\tzstripes[3]{
    \draw[ultra thick,#3] ([yshift=#2]#1.south west) -- ([yshift=#2]#1.south east);
}
\newcommand\tzstripesl[4]{
    \draw[ultra thick,#4] ([yshift=#3]#1.south west) -- ([yshift=#3]#2.south east);
}
\usetikzlibrary{decorations.pathreplacing,calligraphy,arrows.meta}

\newcommand\prename{\texttt{pre}}
\newcommand\postname{\texttt{post}}
\newcommand\modname{\texttt{mod}}


\begin{document}

%% Title information
\title{Your Title}
% \titlenote{with title note}             %% \titlenote is optional;
% \subtitle{Or: Started From the $\bot$\ldots}
% \subtitlenote{with subtitle note}       %% \subtitlenote is optional;

%% Author information
%% Contents and number of authors suppressed with 'anonymous'.
%% Each author should be introduced by \author, followed by
%% \authornote (optional), \orcid (optional), \affiliation, and
%% \email.
%% An author may have multiple affiliations and/or emails; repeat the
%% appropriate command.
%% Many elements are not rendered, but should be provided for metadata
%% extraction tools.

%% Author with single affiliation.
\author{FirstName LastName}
% \authornote{with author1 note}          %% \authornote is optional;
%                                         %% can be repeated if necessary
% \orcid{nnnn-nnnn-nnnn-nnnn}             %% \orcid is optional
% \affiliation{
%   \position{Position1}
%   \department{Department1}              %% \department is recommended
%   \institution{Institution1}            %% \institution is required
%   \streetaddress{Street1 Address1}
%   \city{City1}
%   \state{State1}
%   \postcode{Post-Code1}
%   \country{Country1}                    %% \country is recommended
% }
% \email{first1.last1@inst1.edu}          %% \email is recommended
% 
% %% Author with two affiliations and emails.
% \author{First2 Last2}
% \authornote{with author2 note}          %% \authornote is optional;
%                                         %% can be repeated if necessary
% \orcid{nnnn-nnnn-nnnn-nnnn}             %% \orcid is optional
% \affiliation{
%   \position{Position2a}
%   \department{Department2a}             %% \department is recommended
%   \institution{Institution2a}           %% \institution is required
%   \streetaddress{Street2a Address2a}
%   \city{City2a}
%   \state{State2a}
%   \postcode{Post-Code2a}
%   \country{Country2a}                   %% \country is recommended
% }
% \email{first2.last2@inst2a.com}         %% \email is recommended
% \affiliation{
%   \position{Position2b}
%   \department{Department2b}             %% \department is recommended
%   \institution{Institution2b}           %% \institution is required
%   \streetaddress{Street3b Address2b}
%   \city{City2b}
%   \state{State2b}
%   \postcode{Post-Code2b}
%   \country{Country2b}                   %% \country is recommended
% }
% \email{first2.last2@inst2b.org}         %% \email is recommended


%% Abstract
%% Note: \begin{abstract}...\end{abstract} environment must come
%% before \maketitle command
\begin{abstract}
    Your abstract goes here.
    %
    Super fun!
\end{abstract}


%% 2012 ACM Computing Classification System (CSS) concepts
%% Generate at 'http://dl.acm.org/ccs/ccs.cfm'.
\begin{CCSXML}
<ccs2012>
<concept>
<concept_id>10011007.10011006.10011008</concept_id>
<concept_desc>Software and its engineering~General programming languages</concept_desc>
<concept_significance>500</concept_significance>
</concept>
<concept>
<concept_id>10003456.10003457.10003521.10003525</concept_id>
<concept_desc>Social and professional topics~History of programming languages</concept_desc>
<concept_significance>300</concept_significance>
</concept>
</ccs2012>
\end{CCSXML}

\ccsdesc[500]{Software and its engineering~General programming languages}
\ccsdesc[300]{Social and professional topics~History of programming languages}
%% End of generated code


%% Keywords
%% comma separated list
% \keywords{keyword1, keyword2, keyword3}  %% \keywords are mandatory in final camera-ready submission


%% \maketitle
%% Note: \maketitle command must come after title commands, author
%% commands, abstract environment, Computing Classification System
%% environment and commands, and keywords command.
\maketitle

\section{Introduction}
\label{sec:Introduction}
Your great introduction goes here!

\section{Implementation}
\label{sec:Implementation}
Some discussion about your implementation can go here

\section{Limitations}
\label{sec:Limitations}
If you want to, you can have a limitations section.

\section{Future Work}
\label{sec:FutureWork}
Here you can talk about future work.

\section{Related Work}
\label{sec:Related}
Here you can talk about related work.

\section{Conclusion}
\label{sec:Conclusion}
Here you can conclude.


% %% Acknowledgments
\begin{acks}                            %% acks environment is optional
                                        %% contents suppressed with 'anonymous'
  %% Commands \grantsponsor{<sponsorID>}{<name>}{<url>} and
  %% \grantnum[<url>]{<sponsorID>}{<number>} should be used to
  %% acknowledge financial support and will be used by metadata
  %% extraction tools.
    Here you can acknowledge your acks.
\end{acks}


% %% Bibliography
\bibliography{main}
% 
% 
% %% Appendix
% \appendix
% \section{Appendix}

% Text of appendix \ldots

\end{document}
