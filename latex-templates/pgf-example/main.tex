\documentclass[acmsmall]{acmart}\settopmatter{}

% PREAMBLE, not too important
\usepackage{datatool}
\usepackage{hyperref}
\usepackage{tikz}
\usepackage{pgfplots}
\usepackage{booktabs}
\usepackage{subcaption}
\usepackage{etoolbox}
\usepackage{prettyref}
\newcommand{\pref}{\prettyref}
\newrefformat{thm}{Theorem~\ref{#1}}
\newrefformat{cor}{Corollary~\ref{#1}}
\newrefformat{lem}{Lemma~\ref{#1}}
\newrefformat{cha}{Chapter~\ref{#1}}
\newrefformat{sec}{Section~\ref{#1}}
\newrefformat{app}{Appendix~\ref{#1}}
\newrefformat{tab}{Table~\ref{#1}}
\newrefformat{fig}{Figure~\ref{#1}}
\newrefformat{alg}{Algorithm~\ref{#1}}
\newrefformat{exa}{Example~\ref{#1}}
\newrefformat{def}{Definition~\ref{#1}}
\newrefformat{li}{Line~\ref{#1}}
\newrefformat{eq}{Equation~\ref{#1}}
\newrefformat{exa}{Example~\ref{#1}}

% Helper to let us select which test case rows we want to display in any given
% bar chart.
% https://tex.stackexchange.com/questions/98003/filter-rows-from-a-table
\pgfplotsset{
    discard if not/.style 2 args={
        x filter/.code={
            \edef\tempa{\thisrow{#1}}
            % https://tex.stackexchange.com/questions/279337/etoolbox-how-to-compare-two-lists
            \edef\tempb{}
            \forcsvlist{\listadd\tempb}{#2}
            % https://mirror.las.iastate.edu/tex-archive/macros/latex/contrib/etoolbox/etoolbox.pdf
            \xifinlist{\tempa}{\tempb}{}{\def\pgfmathresult{inf}}
        }
    }
}

% Load the raw data using DTL, which lets us perform computations on the data
% (e.g., accumulation sums for the time-on-x-axis chart).
\DTLloaddb{shatardata}{shatar.csv}

% Now for each solver (column in shatar.csv) we're going to define a pgf
% function sum[solvername] so that sum[solvername](x) is the count of the numbe
% of test cases that [solvername] can solve in at most x seconds.
\newcommand\declaresummer[2]{
    \pgfmathdeclarefunction{sum#1}{1}{%
        \pgfmathsetmacro\ret{0}%
        \DTLforeach{shatardata}{\time={#2}}{%
            % https://tex.stackexchange.com/questions/46232/how-to-check-if-a-macro-expands-to-something-greater-than-a-constant
            \pgfmathsetmacro\ret{\ret + (\time < ##1 ? 1 : 0)}%
        }%
        \pgfmathparse{\ret}%
    }
}

\declaresummer{cgmp}{C GMP}
\declaresummer{rustgmp}{Rust GMP}
\declaresummer{zaleski}{Zaleski Solver}
% Add to this list if you want to support others.
% There's probably a better way of doing this ...

\begin{document}

\title{Hello Test PGF}

\author{FirstName LastName}

\begin{abstract}
    Your abstract goes here.
    %
    Super fun!
\end{abstract}


\maketitle

\section{PGF Examples}
Here are some examples of using PGF/PGFPlots to build plots in LaTeX directly
from your raw CSV data. I've copied some links to useful references into the
LaTeX source file, the most useful are the PGFPlots manual and the PGF/TikZ
manual.

% http://vonbuhren.free.fr/Liens/pgfplots.pdf
% https://pgf-tikz.github.io/pgf/pgfmanual.pdf

The first plot, as shown in~\pref{fig:shatar-rust}, demonstrates how to make a
bar plot. The second plot, shown in~\pref{fig:shatar-accum}, shows how to make
a plot that shows, for each time, how many test cases were solved by that time
for a few solvers.

\LaTeX{} is a pretty wacky programming language (whitespace is \emph{very}
important!) but at the end of the day it is a programming language and you can
do some pretty cool stuff with it!

\begin{figure}[b]
    \begin{subfigure}[t]{0.49\linewidth}
        \centering
        \begin{tikzpicture}[scale=0.75]
            \begin{axis}[
                legend pos=north west,
                symbolic x coords={100v150c,200v300c,300v450c,500v750c,100v200c,200v400c,300v600c,500v1000c,p-1,p-2,p-3,5,7,30v30c,30v35c,30v40c,30v45c,30v50c,80v30c,90v30c},
                ymin=0,
                ymode=log,
                % https://tex.stackexchange.com/questions/86680/pgfplots-and-log-scale-bars-starting-from-the-bottom-of-the-screen
                log origin=infty,
                xtick=data,
                xlabel={Test Case},
                ylabel={Time (Seconds)},
                ybar,
                unbounded coords=jump,
                legend entries={Rust GMP, Rust Bignum},
            ]
                \addplot[fill=blue,discard if not={Test Case}{100v150c,200v300c,300v450c,500v750c}]
                    table[col sep=comma,y=Rust GMP]
                    {shatar.csv};
                \addplot[fill=red,discard if not={Test Case}{100v150c,200v300c,300v450c,500v750c}]
                    table[col sep=comma,y=Rust Bignum]
                    {shatar.csv};
            \end{axis}
        \end{tikzpicture}
        \caption{Some description of your figure. Timeout: 120 seconds.}
        \label{fig:shatar-rust}
    \end{subfigure}
    \begin{subfigure}[t]{0.49\linewidth}
        \centering
        \begin{tikzpicture}[scale=0.75]
            \begin{axis}[
                xmin=0,
                xlabel={Time (Seconds)},
                ylabel={Number of Test Cases Solved},
                domain=0:6,
                samples=20,
                legend entries={C GMP, Rust GMP, Zaleski},
                legend pos=south east,
            ]
                \addplot[blue] {sumcgmp(x)};
                \addplot[red] {sumrustgmp(x)};
                \addplot[orange] {sumzaleski(x)};
            \end{axis}
        \end{tikzpicture}
        \caption{Time vs. N-Solved. Description of your figure.}
        \label{fig:shatar-accum}
    \end{subfigure}
    \caption{Your results. Overall description.}
    \label{fig:shatar-eval}
\end{figure}

\end{document}
